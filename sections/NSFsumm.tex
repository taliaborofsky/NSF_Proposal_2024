%%%%%%%%% SUMMARY -- 1 page, third person
% e.g:  "The PI will prove" not "I will prove"

\section{Project Summary}
\subsection{Overview}
\begin{itemize} 
\item Despite the prominence of cooperatively hunting species in conservation efforts (e.g. lions and wolves), and their recognition in structuring communities, the ecology of their cooperative hunting behavior is not well understood.
\item The project will answer the following questions using two models and a meta-analysis: (1) Can availability of profitable big prey, combined with competition for less profitable small prey, drive the formation of groups of predators? (2) If predators form groups, do they then drive the prey to extinction?, (3) Does reproductive skew limit or encourage the evolution of cooperative hunting?, and (4) Does the spread of cooperative hunting also select for reproductive skew?
\end{itemize}
\subsection{Intellectual Merit}
\begin{itemize}
\item Previous work on evolution of social traits in group-structured populations did not allow enough flexibility in group membership. The P.I. proposes a group-structured model with analytical, continuous time, dynamics in which predators can leave and join groups.  A model of group formation originally developed for studying the coagulation and fragmentation of particles in chemistry was proposed as a model of animal group formation 1994 (Gueron and Levin) which allowed fission-fusion dynamics of groups, but has not been extended to to the study of the ecology and evolution of social behavior until now . 
\item Previous predator-prey models where predators hunted in groups used allee effects for cooperation and did not account for both the impact of sharing and the benefits of group size on the ability of predators to capture larger types of prey, which the proposed models will address.
\item Applying an ecological, niche-construction viewpoint to the study of different cooperative behaviors, which may help explain situations in which cooperation evolves even between group members who are not very related.
\end{itemize}
% This is why your project is interesting and will help further
% knowledge in the field of mathematics. 

\subsection{Broader Impacts}
\begin{itemize}
\item Conservation needs better understanding of social behavior of predators
\item Throughout the meta-analysis, the P.I. will mentor high school students
\end{itemize}
